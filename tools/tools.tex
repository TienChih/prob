\documentclass{ximera}
      
\title{Tools}
      
\begin{document}
      
\begin{abstract}
      
Some Useful Tools.
      
\end{abstract}
      
\maketitle
       
\begin{enumerate} 
\item Mean, Median, Mode, and Standard Deviation and Quartiles:

\begin{sageCell}
L=[1,1,2,3,4,5, 1, 1] # Input the list of Data Here:

L=sorted(L); L


print('The mean is: '+`mean(L)`)
print('The median is: '+`median(L)`)
print('The mode(s) is (are): '+`mode(L)`)
print('The standard deviation is: '+`std(L)`)

Z=(len(L))/2
ZI=floor(Z)

firsthalf=[]
secondhalf=[]

for i in range(ZI):
    firsthalf.append(L[i])

    
for j in range(ZI):
    secondhalf.append(L[len(L)-j-1])


print('Q1 is: '+`median(firsthalf)`)
print('Q3 is: '+`median(secondhalf)`)
\end{sageCell} 
 Enter the list of data as L. 
 
 \item Expected Value:
 
 \begin{sageCell}
 List=[(0,.35), (1,.15), (2,.45), (3,.05)] #Enter as a list of pairs each (x, p(x))

dummy=0
totalprob=0

for i in range(len(List)):
    totalprob=totalprob+List[i][1]
    if List[i][1]<0 or List[i][1]>1:
        dummy=1

if dummy==1 or totalprob!=1:
    print('This does not represent a probability distribution!')
else:
    EVX=0
    EVXs=0
    for i in range(len(List)):
        EVX=EVX+List[i][0]*List[i][1]
        EVXs=EVXs+List[i][0]^2*List[i][1]
    Var=EVXs-EVX^2
    std=(Var)^(.5)
    print('The expected value is '+`EVX`)
    print('The variance is '+`Var`)
    print('The standard deviation is '+`std`)
 \end{sageCell}
 Enter the discrete distribution as pairs $(x, P(X=x))$. 
 
\item Binomial Distributions:

\begin{sageCell}
n = 5 #Enter the number of trials here
p = .7 #Enter

binomdist=[]

for i in range(n+1):
    binomdist.append((i, binomial(n,i)*p^i*(1-p)^(n-i)))

chartlist=[]
for i in range(n+1):
    chartlist.append(binomdist[i][1])
    

bar_chart(chartlist)    
    
print('The binomial distribution with '+`n`+' trials each with probability of success '+`p`+' is:')
binomdist
print('The expected value is: '+`n*p`)
print('with variance '+`n*p*(1-p)`)
print('and standard deviation '+`(n*p*(1-p))^(.5)`)
\end{sageCell}
Enter in the number of trials $n$ and the probability of success per trial $p$.  If the code is improperly indented, copy and paste: 



 \item Normal Curves
 
 \begin{onlineOnly}
$$\graph[xmin=-2,xmax=2,ymin=-0.5,ymax=1, panel]{N=e^{-x^2/2}/(2\pi)^{(0.5)}, P=\int_{z_1}^{z_2} N dx, a=-1, b=1, s=1, m=0, z_1=(a-m)/s, z_2=(b-m)/s, 0<y<N\left\{z_1<x<z_2\right\}}$$
\end{onlineOnly}

\begin{sageCell}
from scipy.stats import norm

normdist = norm()

mean=120 # Put the mean of the distribution here
std=30  # Put the standard deviation of the distribution here.
P=0.4  # P is the value such that everything to the left of X is P

X=normdist.ppf(P)*std+mean

print('The area to the left of '+`X`); print(' on the normal curve with mean '+`mean`); print(' and standard deviation '+`std`); print(' is  '+`P`)
\end{sageCell}

\end{enumerate}

 
      






\end{document}
